\section{Continuous Random Variable}
\begin{definition}
    Let $X$ denote any random variable. The \emph{distribution function} of $X$, denoted by $F(x)$
    is such that $F(x) = P(X \leq x)$ for $-\infty < x < \infty$. And $X$ is continuous random variable iff. $F(X)$ is continuous for $-\infty < x < \infty$ 
\end{definition}


\begin{definition}
    Let $X$ denote any continuous random variable with distribution function $F(x)$.
    If this derivative exists

    \[
    f(x) = \frac{d}{dx}F(x) = F^\prime(x)    
    \],
    then $f(x)$ is called the \emph{probability density function} for the continuous random variable $X$
\end{definition}
\begin{corollary*} It follows from the definitions that
    \[F(x) = \int^x_{-\infty} f(x) dx\]
\end{corollary*}
\begin{theorem}
    If the random variable $X$ with the density function $f(x)$ and $a < b$, 
    then
    \[
    P(a\leq x \leq b) = \int_a^b f(x) dx    
    \]
\end{theorem}

\begin{definition}[Expected Value]\label{def:exp_val_cont}
    Let $X$ be a continuous random variable with the density function $f(x)$. Then the \emph{expected value} of $X$, $E(X)$, is defined to be
    \[
        E(X) = \int_{-\infty}^{\infty} xf(x) dx
    \]
\end{definition}

\section{Uniform Probability distribution}


\begin{definition}
    For $\theta_1 < \theta_2$, a random vairable $X$ has a continuous \emph{uniform distribution} on the interval $(\theta_1, \theta_2)$ 
    iff. the density function of $X$ is
    \[ 
        f(x) = 
        \begin{cases} 
            \frac{1}{\theta_2 - \theta_1} &, \theta_1 < x <\theta_2, \\
            0 &, \text{elsewhere}.
        \end{cases}
    \]
\end{definition}

\begin{theorem}[mean and variance of uniform distribution]
    For $\theta_1 < \theta_2$ and $X$ is a continuous random variable on the interval $(\theta_1, \theta_2)$,
    then

    \[
    E(X) = \frac{\theta_1 + \theta_2}{2} \text{ and } V(X) = \frac{{(\theta_2 - \theta_1)}^2}{12}
    \]
\end{theorem}
\begin{proof}
    The proof is left as exercises for readers. (common phrases in math textbooks)
\end{proof}
